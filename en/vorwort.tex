\cleardoublepage
\thispagestyle{empty}
\begin{center}
	
	{\Large\textbf{Photon – Theory and Applications}}\\[1.2em]
	{\large Dipl.-Ing.\,(FH)\,Christian Weilharter}\\[1.2em]
	\textcopyright~2025, Christian Weilharter, Traunstein\\[2em]
\end{center}

\begin{flushleft}
	\begin{tabular}{@{}l l}
	
		\textbf{ISBN (E-Book):} & 978-3-912302-03-5 \\[0.5em]
		\textbf{Edition:} & 1st edition, 2025 \\[0.5em]
		\textbf{Typesetting:} & \LaTeX \\[0.5em]
		\textbf{Publisher:} & Self-published by Christian Weilharter \\[0.5em]


	
		\textbf{E-book edition:} & Apple Books \\[0.5em]
		& Amazon Kindle Direct Publishing \\[0.5em]
	
\textbf{Contact:} & \href{mailto:info@mathandphysics.eu}{info@mathandphysics.eu}\\[0.5em]
\textbf{Web:} & \href{https://www.mathandphysics.eu}{www.mathandphysics.eu}\\

	\end{tabular}
\end{flushleft}

\vspace{2em}
\noindent
All rights reserved. No part of this book may be reproduced, stored, or transmitted
in any form or by any means, electronic or mechanical, including photocopying,
recording, or by any information storage and retrieval system, without prior written
permission of the author.

\begin{center}\small Printed in Germany\end{center}

\cleardoublepage


\chapter*{Preface}
\markboth{Preface}{Preface} % optional für Kopfzeile
% kein addcontentsline → Preface erscheint NICHT im Inhaltsverzeichnis



The creation of this book has been driven by a deep fascination with light and its fundamental mediator – the photon. In modern physics, the photon plays a central role: as a particle without mass, yet carrying energy and momentum; as the messenger of the electromagnetic interaction; and as a key figure in quantum mechanics.  
\noindent
My goal has been to present this multifaceted concept in its historical development, its physical significance, and its technical applications in such a way that both interested laypersons and advanced readers gain a solid understanding – without unnecessary oversimplification, yet always in a clear and accessible manner.  
\noindent
In the conception and preparation of this book, modern technology was also employed. The AI system ChatGPT by OpenAI (as of 2025) was used in a supportive role for formulating, structuring, and reflecting on the content. This collaboration made it possible to articulate ideas more quickly, test alternative phrasings, and clarify complex relationships.  
All content, however, was critically reviewed, revised, and is solely the responsibility of the author.  
\noindent
Thus, this book is not only a contribution to the didactics of modern physics – it is also an experiment in how traditional science can gain a new level of clarity and accessibility through modern tools.  
\noindent
I hope that the enthusiasm for this subject will resonate with the readers – just as it has accompanied me for many years.  

\begin{flushright}
	\textit{Dipl.-Ing.(FH) Christian Weilharter} \\
	\vspace{0.5em}
	[Traunstein, 2025]
\end{flushright}
