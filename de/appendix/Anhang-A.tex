

\appendix
\renewcommand{\thechapter}{A}

\renewcommand{\thesection}{\Alph{chapter}.\arabic{section}}

\chapter{Mathematische Hintergründe und Herleitungen}
\label{anhangA}

In diesem Anhang werden die im Haupttext angesprochenen physikalischen Konzepte
formal und mathematisch vertieft. Ziel ist es, die didaktische Lesbarkeit der
Kapitel nicht zu beeinträchtigen und zugleich interessierten Lesern die
vollständigen Herleitungen zugänglich zu machen. 

Die Abschnitte sind thematisch nach den zentralen Eigenschaften des Photons
gegliedert, darunter Energie-Impuls-Relation, Massehypothese, Helizität und
Polarisation.\index{Photon!Eigenschaften} Auf diese Weise bildet der Anhang eine Brücke zwischen den
intuitiven Erklärungen im Haupttext und der mathematischen Strenge der
Quantenfeldtheorie.

%\section{Kapitel I}
\phantomsection
\section{Energie- und Impulsrelation des Photons}\index{Energie-Impuls-Relation!Photon}
\label{anhangA:energie_impuls}

In diesem Abschnitt wird formal hergeleitet, warum ein Photon die Energie
\[
E = h f
\]
und den Impuls
\[
p = \frac{h}{\lambda}
\]
besitzt. Ausgangspunkt sind die Maxwell-Gleichungen und deren Wellengleichung.
Über den Poynting-Vektor und die Energiedichte des elektromagnetischen Feldes
wird gezeigt, dass die Quantisierung der Felder zu diskreten Energieportionen
führt. Diese Herleitung ergänzt die intuitive Darstellung im Haupttext
(Kapitel~I).

\section{Das elektromagnetische Feld in der relativistischen Formulierung}\index{Elektromagnetisches Feld!relativistische Formulierung}\index{Feldstärketensor}\index{Lagrangedichte}
\label{anhangA:feldtheorie}

Hier erfolgt die Einführung in die formale Beschreibung des elektromagnetischen
Feldes:

\begin{itemize}
	\item Viererpotential $A^\mu = (\phi, \vec{A})$
	\item Feldstärketensor $F^{\mu\nu} = \partial^\mu A^\nu - \partial^\nu A^\mu$
	\item Antisymmetrie-Eigenschaft $F^{\mu\nu} = -F^{\nu\mu}$
	\item Lagrangedichte
	\[
	\mathcal{L} = -\frac{1}{4} F_{\mu\nu}F^{\mu\nu}
	\]
	\item Zusammenhang zur Quantisierung: Photon als Eichboson des $U(1)$
	Elektromagnetismus
\end{itemize}

Dies liefert den mathematischen Unterbau zur Aussage im Haupttext, dass
Photonen „Anregungen des elektromagnetischen Feldes“ sind.

\section{Formale Beschreibung verschränkter Photonen}\index{Photonen!Verschränkung}\index{Dirac-Notation}\index{SPDC (Spontane parametrische Fluoreszenz)}
\label{anhangA:verschr}

Die im Haupttext (Kapitel~I) dargestellten Experimente zur Erzeugung
verschränkter Photonenpaare (SPDC) lassen sich formal in der
Dirac-Notation beschreiben. Ein typisches verschränktes
Polarisations-Zustandspaar lautet:

\[
|\psi\rangle = \frac{1}{\sqrt{2}}\big( |H\rangle_A \otimes |V\rangle_B +
|V\rangle_A \otimes |H\rangle_B \big)
\]

\begin{itemize}
	\item $|H\rangle$: horizontal polarisierter Zustand
	\item $|V\rangle$: vertikal polarisierter Zustand
	\item Indizes $A, B$: die beiden Photonen
\end{itemize}

Dieser Formalismus macht die im Experiment beobachteten Korrelationen
transparent und zeigt, warum klassische Modelle mit verborgenen Variablen
nicht ausreichen.

%Kapitel II
%\section{Kapitel II}
\section{Herleitung des Rayleigh-Jeans-Gesetzes}\index{Rayleigh-Jeans-Gesetz!Herleitung}\index{Rayleigh, Lord}\index{Jeans, James}\index{Ultraviolett-Katastrophe}
\label{anhangA:rayleigh}

Das Rayleigh-Jeans-Gesetz ergibt sich, wenn man die elektromagnetischen
Moden in einem Hohlraum wie harmonische Oszillatoren behandelt:

\begin{enumerate}
	\item Man zählt die Anzahl der stehenden Wellen im Würfelvolumen $V$.
	\item Jede Mode besitzt zwei Polarisationsrichtungen.
	\item Nach dem Ausrüstungsprinzip der klassischen Thermodynamik trägt
	jede Freiheitsgrad im Gleichgewicht die mittlere Energie $kT$.
\end{enumerate}

Die Anzahl der Moden zwischen den Frequenzen $\nu$ und $\nu + d\nu$ ist
\[
g(\nu)\, d\nu = \frac{8\pi V \nu^2}{c^3}\, d\nu.
\]

Multipliziert mit $kT$ ergibt dies die spektrale Energiedichte
\[
u(\nu, T) = \frac{8\pi \nu^2}{c^3}\, kT,
\]
die in Wellenlängenform $u(\lambda, T) = \tfrac{8\pi kT}{\lambda^4}$ lautet.
Das Gesetz stimmt bei langen Wellenlängen, divergiert jedoch für
$\lambda \to 0$ – die \emph{Ultraviolett-Katastrophe}.

\section{Das Wiensche Strahlungsgesetz}\index{Wiensches Strahlungsgesetz}\index{Wien, Wilhelm}
\label{anhangA:wien}

Wilhelm Wien leitete 1896 eine Näherung für die Schwarzkörperstrahlung her.
Die Argumentation beruhte auf:

\begin{itemize}
	\item Thermodynamischen Überlegungen: Adiabatische Kompression eines
	Hohlraums verschiebt das Strahlungsspektrum.
	\item Dimensionsanalyse: Die Intensität hängt von $T$ und $\lambda$ ab
	und muss die korrekten Einheiten haben.
\end{itemize}

Das Resultat war
\[
u(\lambda, T) = \frac{c_1}{\lambda^5}\,
\exp\!\left(-\frac{c_2}{\lambda T}\right),
\]
mit Konstanten $c_1, c_2$, die erst durch Plancks Ansatz vollständig
verstanden wurden. Das Wiensche Gesetz stimmt im UV-Bereich,
versagt jedoch bei langen Wellenlängen.

\section{Herleitung des Planckschen Strahlungsgesetzes}\index{Plancksches Strahlungsgesetz!Herleitung}\index{Planck, Max}
\label{anhangA:planck}

Planck kombinierte die Grenzgesetze von Rayleigh-Jeans und Wien und
führte die Quantisierung der Energie ein:

\begin{itemize}
	\item Ein Oszillator kann nur Energien $E_n = nh\nu$ annehmen.
	\item Die Besetzungswahrscheinlichkeit folgt der Boltzmann-Verteilung.
\end{itemize}

Die mittlere Energie pro Oszillator lautet
\[
\langle E \rangle = \frac{h\nu}{e^{h\nu/kT} - 1}.
\]

Multipliziert mit der Modenzahl
$g(\nu)\, d\nu = \tfrac{8\pi V \nu^2}{c^3}\, d\nu$
ergibt dies die Strahlungsdichte
\[
u(\nu, T) = \frac{8\pi \nu^2}{c^3}\,
\frac{h\nu}{e^{h\nu/kT} - 1}.
\]

Dies ist das \textbf{Plancksche Strahlungsgesetz}, das für alle Frequenzen
mit dem Experiment übereinstimmt.

\section{Mathematische Beschreibung des Photoeffekts}\index{Photoeffekt!Mathematische Beschreibung}
\label{anhangA:photoeffekt}

Beim Photoeffekt absorbiert ein Elektron in einem Metall ein Photon
mit Energie $E_\gamma = h\nu$. Um das Elektron aus dem Metall zu lösen,
muss die \emph{Austrittsarbeit} $A$ überwunden werden. Die überschüssige
Energie geht in die kinetische Energie des Elektrons über:

\[
E_\text{kin} = h\nu - A.
\]

Daraus folgt eine \emph{Grenzfrequenz}
\[
\nu_\text{min} = \frac{A}{h},
\]
unterhalb derer keine Elektronen ausgelöst werden – unabhängig von der
Lichtintensität. Diese lineare Beziehung zwischen Elektronenenergie und
Lichtfrequenz wurde in Millikans Experimenten (1916) präzise bestätigt.\index{Millikan, Robert A.}
%\section{Kapitel III}
\section{Photonenimpuls}\index{Photon!Impuls}\index{Energie-Impuls-Relation}
\label{anhangA:impuls}

Der Impuls eines Photons lässt sich aus der relativistischen Energie-Impuls-Relation ableiten. Für beliebige Teilchen gilt
\[
E^2 = (pc)^2 + (m_0 c^2)^2 ,
\]
wobei $E$ die Energie, $p$ der Impuls, $c$ die Lichtgeschwindigkeit und $m_0$ die Ruhemasse ist.

\begin{itemize}
	\item Für masselose Teilchen ($m_0 = 0$) reduziert sich diese Gleichung zu
	\[
	E = p c.
	\]
	
	\item Für das Photon gilt zugleich die Quantisierungsbedingung
	\[
	E = h f = \frac{h c}{\lambda}.
	\]
	
	\item Setzt man beide Ausdrücke für $E$ gleich, folgt unmittelbar
	\[
	p = \frac{E}{c} = \frac{h}{\lambda}.
	\]
\end{itemize}

\noindent
Damit ist der Impuls eines Photons direkt an seine Wellenlänge gekoppelt. Diese Beziehung ist eine der zentralen Brücken zwischen Wellen- und Teilchenbeschreibung des Lichts.

\section{Energie-Impuls-Relation des Photons}\index{Photon!Massehypothese}\index{Photon!Masselosigkeit}\index{Energie-Impuls-Relation}
\label{anhangA:masse}

Für relativistische Teilchen gilt allgemein die Energie-Impuls-Relation:
\[
E^2 = (pc)^2 + (m c^2)^2.
\]

\begin{itemize}
	\item \textbf{Masseloses Photon:}  
	Setzt man $m=0$, so folgt unmittelbar:
	\[
	E = p c.
	\]
	Dies steht im Einklang mit den Relationen $E = h f$ und $p = h/\lambda$.
	
	\item \textbf{Hypothetisch massives Photon:}  
	Angenommen, das Photon hätte eine Ruhemasse $m_\gamma \neq 0$, so ergäbe sich:
	\[
	E^2 = (p c)^2 + (m_\gamma c^2)^2.
	\]
	Ein solches Photon würde sich stets langsamer als $c$ bewegen, und die Lichtgeschwindigkeit wäre nicht mehr universell konstant. 
	Bereits minimale Abweichungen von $m=0$ würden sich in Präzisionsexperimenten zeigen.
\end{itemize}

Experimentell konnte bisher nur eine obere Schranke für die Photonenmasse gesetzt werden. Aktuelle Grenzen liegen bei
\[
m_\gamma < 10^{-18}\,\text{eV}/c^2,
\]
was faktisch bedeutet, dass das Photon als masselos angesehen wird.

\section{Helizität des Photons}\index{Photon!Helizität}\index{Spin}
\label{anhangA:helizitaet}

Das Photon besitzt Spin $s=1$, doch aufgrund seiner Masselosigkeit sind nicht alle drei Spinprojektionen ($m_s=-1,0,+1$) physikalisch realisierbar. 

\begin{itemize}
	\item \textbf{Allgemeiner Spin-1-Zustand:}  
	Für massive Spin-1-Teilchen sind drei Polarisationszustände möglich, entsprechend den Projektionen $m_s=-1,0,+1$ auf die Bewegungsrichtung.
	
	\item \textbf{Masseloses Photon:}  
	Da das Photon keine Ruhemasse hat, existiert kein Ruhesystem, in dem man die Spinausrichtung unabhängig vom Bewegungsvektor definieren könnte.  
	Mathematisch erzwingt die Eichinvarianz der Maxwellgleichungen (bzw. des QED-Formalismus), dass die longitudinale Komponente ($m_s=0$) verschwindet.
	
	\item \textbf{Helizitätszustände:}  
	Übrig bleiben nur zwei mögliche Zustände:
	\begin{align*}
		\lambda& = +1 \quad \text{(Rechtsdrehung, Rechtspolarisation)}\\
		\lambda &= -1 \quad \text{(Linksdrehung, Linkspolarisation)}
	\end{align*}
	
	Diese werden als die beiden Helizitätszustände des Photons bezeichnet.
\end{itemize}

\noindent
Damit ist das Photon ein zweizuständiges masseloses Boson, dessen Freiheitsgrade vollständig durch die beiden möglichen Helizitäten beschrieben werden.

\section{Polarisation des Photons}\index{Photon!Polarisation}\index{Dirac-Notation}\index{Jones-Vektor}
\label{anhangA:polarisation}

Die Polarisation beschreibt die transversale Schwingungsrichtung des elektrischen Feldes eines Photons. Formal lässt sich dieser Freiheitsgrad auf zwei Arten darstellen:

\begin{itemize}
	\item \textbf{Dirac-Notation:}  
	In der Quantenmechanik werden Polarisationszustände als Basisvektoren in einem zweidimensionalen Hilbertraum notiert:  
	\[
	|H\rangle = \begin{pmatrix}1 \\ 0\end{pmatrix}, \qquad
	|V\rangle = \begin{pmatrix}0 \\ 1\end{pmatrix},
	\]
	wobei $|H\rangle$ für horizontale und $|V\rangle$ für vertikale Polarisation steht.  
	Beliebige Polarisationszustände lassen sich als Linearkombination schreiben:  
	\[
	|\psi\rangle = \alpha |H\rangle + \beta |V\rangle, \quad |\alpha|^2 + |\beta|^2 = 1.
	\]
	
	\item \textbf{Jones-Vektoren:}  
	In der klassischen Optik wird derselbe Zustand durch \emph{Jones-Vektoren} beschrieben:  
	\[
	\vec{E} = \begin{pmatrix} E_x \\ E_y \end{pmatrix},
	\]
	wobei $E_x$ und $E_y$ komplexe Amplituden der elektrischen Feldkomponenten in $x$- und $y$-Richtung sind.  
	Auch hier gilt: normiert man die Intensität, so entspricht dies der Normierung des Zustandsvektors in der Dirac-Notation.
\end{itemize}

\noindent
Beide Beschreibungen sind äquivalent – die Dirac-Notation betont den quantenmechanischen Zustandsraum, während die Jones-Vektoren die klassische elektromagnetische Wellenoptik widerspiegeln.  
Die Verknüpfung dieser Darstellungen ist in der Quantenoptik ein zentrales Werkzeug.
%\section{Kapitel IV}
\section{Herleitung der Einstein-Gleichung beim Photoeffekt}
\label{anhangA:photoeffekt}

Die Einstein-Gleichung\index{Einstein-Gleichung}\index{Photoeffekt}
\[
E_{\text{kin}} = h \nu - A
\]
ergibt sich aus einer einfachen Energiebilanz zwischen Photon\index{Photon} und Elektron\index{Elektron}.

\begin{enumerate}
	\item Ein Photon besitzt die Energie
	\[
	E_{\text{Photon}} = h \nu,
	\]
	wobei \( h \) das Plancksche Wirkungsquantum\index{Plancksches Wirkungsquantum}\index{Planck, Max} und \( \nu \) die Frequenz\index{Frequenz} des einfallenden Lichts ist.
	
	\item Um ein Elektron aus dem Metall zu lösen, muss die \textbf{Austrittsarbeit} \( A \)\index{Austrittsarbeit} überwunden werden. Diese entspricht der minimalen Bindungsenergie der Elektronen im Festkörper.
	
	\item Bleibt nach Überwindung von \( A \) Energie übrig, so erscheint sie als kinetische Energie\index{Kinetische Energie} des Elektrons:
	\[
	E_{\text{kin}} = E_{\text{Photon}} - A.
	\]
	
	\item Damit folgt direkt:
	\[
	E_{\text{kin}} = h \nu - A.
	\]
\end{enumerate}

\textbf{Bemerkung:}  
Wird zusätzlich eine Gegenspannung \( U \)\index{Gegenspannung} angelegt, so gilt
\[
eU = h \nu - A,
\]
wobei \( e \) die Elementarladung\index{Elementarladung} ist. Diese Form erlaubt eine direkte experimentelle Bestimmung von \( h \) durch Messung der Stoppspannung\index{Stoppspannung} als Funktion der Frequenz.

\section{Planck–Einstein-Beziehung \texorpdfstring{$E = h\nu$}{E = hν}}
\label{anhangA:planckEinstein}
\index{Planck–Einstein-Beziehung}

\textbf{Ziel.} Wir begründen, warum ein einzelnes Lichtquant (Photon) die Energie
\[
E = h\nu = \hbar \omega
\]
trägt.

\subsection*{Weg 1: Quantisierung der Normalmoden des elektromagnetischen Feldes.}
\phantomsection
Das freie elektromagnetische Feld\index{Elektromagnetisches Feld} in einem Volumen \(V\) lässt sich in ebene Normalmoden mit Kreisfrequenzen \(\omega_{\mathbf{k}}\) zerlegen. Jede Mode entspricht einem harmonischen Oszillator\index{Harmonischer Oszillator} mit Hamiltonoperator
\[
\hat H_{\mathbf{k}}=\hbar \omega_{\mathbf{k}}\!\left(\hat a_{\mathbf{k}}^\dagger \hat a_{\mathbf{k}}+\tfrac12\right).
\]
Die Eigenwerte sind \((n+\tfrac12)\hbar\omega_{\mathbf{k}}\), \(n\in \mathbb{N}_0\). 
Eine Anregung \(\Delta n = 1\) erhöht die Energie exakt um \(\Delta E = \hbar \omega\). 
Diese Energiezunahme identifizieren wir mit \emph{einem Photon} in dieser Mode:
\[
E_{\text{Photon}}=\hbar\omega=h\nu.
\]

\subsection*{Weg 2: Von Plancks Quantenhypothese zu Einsteins Lichtquant.}
\phantomsection
Planck\index{Planck, Max} postulierte 1900 für Materieoszillatoren diskrete Energien \(E_n=n h\nu\).
Einstein\index{Einstein, Albert} übertrug 1905 die Quantisierung auf das \emph{Strahlungsfeld} selbst: die Energie der Strahlung verhält sich, als wäre sie in räumlich lokalisierte \emph{Energiepakete} der Größe \(h\nu\) gebündelt. 
Nur so lassen sich u.\,a. Entropieeigenschaften Wiensches Strahlungsgesetz\index{Wiensches Strahlungsgesetz}) und der Photoeffekt konsistent erklären. 
Damit folgt für ein einzelnes Lichtquant:
\[
E_{\text{Photon}} = h\nu.
\]

\subsection*{Konsequenzen.}
\phantomsection
(i) Die Photonenenergie hängt \emph{nur} von der Frequenz ab (nicht von der Intensität\index{Intensität}). 
(ii) In Verbindung mit \(E_{\text{kin}}=h\nu-A\) erklärt dies die Grenzfrequenz\index{Grenzfrequenz} beim Photoeffekt. 
(iii) Zusammen mit der Feldquantisierung führt dies zu Teilchenzahloperator \(\hat N=\hat a^\dagger \hat a\)\index{Teilchenzahloperator} und einer klaren Zuweisung der Energie pro Photon.

\section{Herleitung der Stoppspannungsgleichung}
\label{anhangA:stoppspannung}

\textbf{Ziel.} Zusammenhang zwischen Photonenenergie, Austrittsarbeit und messbarer Gegenspannung \(U\).

\subsection*{Ausgangspunkt.}
\phantomsection
Die Energiebilanz beim Photoeffekt lautet
\[
E_{\text{Photon}} = h\nu = A + E_{\text{kin,max}}.
\]

\subsection*{Experimentelles Prinzip.}
\phantomsection
In einer Fotozelle\index{Fotozelle} wird zwischen Kathode\index{Kathode} und Anode\index{Anode} eine \emph{Gegenspannung} \(U\) angelegt.  
Elektronen mit kinetischer Energie \(E_{\text{kin}}\) müssen Arbeit \(eU\) verrichten, um die Anode zu erreichen.  
Bei der \textbf{Stoppspannung} \(U_0\)\index{Stoppspannung} ist die Energie gerade aufgebraucht:
\[
E_{\text{kin,max}} = eU_0.
\]

\subsection*{Herleitung.}
\phantomsection
Setzt man dies in die Energiebilanz ein:
\[
h\nu = A + eU_0,
\]
folgt unmittelbar:
\[
eU_0 = h\nu - A.
\]

\subsection*{Experimentelle Bedeutung.}
\phantomsection
- Der Graph \(U_0(\nu)\) ist eine Gerade mit Steigung \(h/e\).  
- Der Achsenabschnitt liefert die materialabhängige Austrittsarbeit \(A\).  
- Millikan\index{Millikan, Robert} bestimmte damit 1916 den Wert des Planckschen Wirkungsquantums mit hoher Genauigkeit – und bestätigte Einsteins Hypothese experimentell.

\subsection*{Konsequenz.}
\phantomsection
Die Stoppspannung erlaubt eine direkte Messung fundamentaler Naturkonstanten, ohne dass Intensität oder Anzahl der Photonen eine Rolle spielen.

\section{Die Austrittsarbeit}
\label{anhangA:austrittsarbeit}

Die \textbf{Austrittsarbeit} \( A \) ist die Mindestenergie, die notwendig ist, um ein Elektron aus einem Metall zu lösen. 
Sie hängt vom Material und der elektronischen Struktur der Oberfläche ab. 

\textbf{Formale Definition:}
\[
A = E_{\text{Fermi}} + E_{\text{Bindung}} - E_{\text{Vakuum}}
\]
wobei \( E_{\text{Vakuum}} \) das Energieniveau eines Elektrons im Vakuum darstellt.

\textbf{Typische Werte:}
\begin{itemize}
	\item Alkalimetalle (z.\,B. Cäsium, Kalium)
	\item Übergangsmetalle (z.\,B. Eisen, Kupfer)
	\item Edelmetalle (z.\,B. Platin)
\end{itemize}

Die Austrittsarbeit erklärt, warum nur Photonen oberhalb einer \emph{Grenzfrequenz} \( \nu_0 = A/h \) Elektronen freisetzen können. 
Sie ist eine materialspezifische Größe und kann durch Oberflächenzustand, Temperatur oder Beschichtung variieren.

\section{Herleitung der Compton-Formel}
\label{anhangA:comptonHerleitung}

Die Herleitung der \textbf{Compton-Formel}\index{Compton-Formel} basiert auf Energie- und Impulserhaltung\index{Impulserhaltung} beim Stoß eines Photons mit einem ruhenden Elektron. 

\textbf{Ausgangssituation:}
\begin{itemize}
	\item Ein Photon mit Wellenlänge \( \lambda \) trifft auf ein Elektron in Ruhe.
	\item Nach dem Stoß hat das Photon die Wellenlänge \( \lambda' \) und wird um den Winkel \( \theta \) abgelenkt.
	\item Das Elektron erhält einen Rückstoßimpuls \( \vec{p}_e \).
\end{itemize}

\textbf{Erhaltungssätze:}
\begin{align*}
	E_\gamma + m_e c^2 &= E'_\gamma + E_e \\
	\vec{p}_\gamma &= \vec{p'}_\gamma + \vec{p}_e
\end{align*}
mit
\[
E_\gamma = \frac{hc}{\lambda}, \quad 
E'_\gamma = \frac{hc}{\lambda'}, \quad 
p_\gamma = \frac{h}{\lambda}, \quad 
E_e^2 = (p_e c)^2 + (m_e c^2)^2.
\]

\textbf{Ergebnis:}
\[
\Delta \lambda = \lambda' - \lambda 
= \frac{h}{m_e c}(1 - \cos \theta).
\]

Diese Verschiebung ist unabhängig von der Photonenenergie und hängt nur vom Streuwinkel ab. Der Faktor
\[
\lambda_C = \frac{h}{m_e c} \approx 2{,}43 \cdot 10^{-12}\,\mathrm{m}
\]
wird als \textbf{Compton-Wellenlänge} des Elektrons\index{Compton-Wellenlänge}\index{Compton, Arthur} bezeichnet.

\section{Der Doppelspalt im quantenmechanischen Formalismus}
\label{anhangA:doppelspalt}

Der Doppelspaltversuch\index{Doppelspaltversuch} mit einzelnen Photonen lässt sich nur mit Hilfe der Quantenmechanik\index{Quantenmechanik} verstehen. 
Im Gegensatz zur klassischen Wellentheorie\index{Wellentheorie} oder Teilchenmechanik\index{Teilchenmechanik} betrachtet man die \textbf{Wellenfunktion}\index{Wellenfunktion} 
eines Photons und deren Überlagerung.

\textbf{Superpositionsprinzip:}\index{Superpositionsprinzip}  
Treffen zwei mögliche Wege \( W_1 \) und \( W_2 \) auf, so gilt für die Gesamtamplitude:
\[
\Psi_{\text{gesamt}} = \Psi_{1} + \Psi_{2}.
\]

\textbf{Darstellung in der Dirac-Notation:}\index{Dirac-Notation}  
Sei \(|1\rangle\) der Zustand „Photon geht durch Spalt 1“ und \(|2\rangle\) der Zustand „Photon geht durch Spalt 2“.  
Ohne Messung des Weges gilt:
\[
|\psi\rangle = \frac{1}{\sqrt{2}} \left( |1\rangle + |2\rangle \right).
\]

\section{Antibunching und die Korrelationsfunktion}
\label{anhangA:antibunching}

Das Phänomen des \textbf{Antibunching}\index{Antibunching} zeigt, dass Photonen \emph{einzeln} emittiert werden. 
Mathematisch wird dies mit der Korrelationsfunktion zweiter Ordnung\index{Korrelationsfunktion} beschrieben.

\textbf{Definition:}
\[
g^{(2)}(\tau) = \frac{\langle I(t) \, I(t+\tau) \rangle}{\langle I(t) \rangle^2},
\]
wobei \( I(t) \) die Intensität (bzw. Zählrate) am Detektor ist und \(\tau\) die Zeitverzögerung zwischen zwei Messungen.

\textbf{Antibunching:}
Bei einer idealen Einzelphotonenquelle\index{Einzelphotonenquelle} findet man:
\[
g^{(2)}(0) = 0.
\]

\textbf{Physikalische Konsequenz:}
- Antibunching widerspricht jeder klassischen Wellenvorstellung.  
- Es zeigt die \textbf{Unteilbarkeit des Photons}: entweder wird es hier oder dort nachgewiesen – aber niemals gleichzeitig an zwei Orten.  
- Damit ist Antibunching ein direkter Beweis für die Quantennatur des Lichts.

\section{Hong--Ou--Mandel- Interferenz am Strahlteiler}
\label{anhangA:HOM}

Der \textbf{Hong--Ou--Mandel (HOM)-Effekt}\index{Hong-Ou-Mandel-Effekt}\index{Hong, Chung-ki}\index{Mandel, Leonard} beschreibt die Zwei-Pho\-tonen-Interferenz\index{Zwei-Photonen-Interferenz} identischer Photonen an einem 50:50-Strahlteiler.\index{Strahlteiler}
Bei perfekter Ununterscheidbarkeit\index{Ununterscheidbarkeit von Photonen} verschwinden Koinzidenzen an den beiden Ausgängen (``HOM-Dip''\index{HOM-Dip}).

\subsection*{Strahlteiler-Transformation (Heisenberg-Bild).}\index{Heisenberg-Bild}
\phantomsection
Für die Eingangsmoden \(\hat a,\hat b\) und Ausgangsmoden \(\hat c,\hat d\) eines verlustfreien 50:50-Strahlteilers wählen wir die unitäre Transformation
\[
\begin{pmatrix}
	\hat c \\ \hat d
\end{pmatrix}
= \frac{1}{\sqrt{2}}
\begin{pmatrix}
	1 & i \\
	i & 1
\end{pmatrix}
\begin{pmatrix}
	\hat a \\ \hat b
\end{pmatrix},
\qquad
\begin{pmatrix}
	\hat a \\ \hat b
\end{pmatrix}
= \frac{1}{\sqrt{2}}
\begin{pmatrix}
	1 & -i \\
	-i & 1
\end{pmatrix}
\begin{pmatrix}
	\hat c \\ \hat d
\end{pmatrix}.
\]
Für die Erzeugungsoperatoren gilt entsprechend
\[
\hat a^\dagger=\frac{\hat c^\dagger - i \hat d^\dagger}{\sqrt{2}},
\qquad
\hat b^\dagger=\frac{-i\,\hat c^\dagger + \hat d^\dagger}{\sqrt{2}}.
\]

\subsection*{Eingangszustand und Ausgangszustand.}
\phantomsection
Zwei einzelne Photonen, eins in jeder Eingangs­mode,
\(|\psi_{\text{in}}\rangle=\hat a^\dagger \hat b^\dagger |0\rangle\),
werden zu
\[
|\psi_{\text{out}}\rangle
= \hat a^\dagger \hat b^\dagger |0\rangle
= \frac{1}{2}\,(\hat c^\dagger - i \hat d^\dagger)(-i\,\hat c^\dagger + \hat d^\dagger)\,|0\rangle
= -\frac{i}{2}\!\left(\hat c^{\dagger 2}+\hat d^{\dagger 2}\right)|0\rangle.
\]
Da \(\hat c^\dagger \hat d^\dagger\)-Terme sich exakt auslöschen, enthält der Ausgangszustand \emph{keine} \(|1_c,1_d\rangle\)-Komponente (keine Koinzidenz). Nach Normierung ergibt sich äquivalent
\[
|\psi_{\text{out}}\rangle
\propto |2_c,0_d\rangle \,\pm\, |0_c,2_d\rangle,
\]
wobei das relative Vorzeichen nur von der Phasenkonvention des Strahlteilers abhängt, die Physik (Verschwinden der Koinzidenzen) jedoch identisch bleibt.

\subsection*{Unvollständige Überlappung und HOM-Dip.}
\phantomsection
Reale Photonen sind zeitlich/spektral/polarisatorisch endliche Wellenpakete. Sei
\(\Lambda(\tau)=\int\!dt\, f_a(t)\,f_b^*(t+\tau)\)
die (komplexe) Überlappungsintegral der zeitlichen Moden (Verzögerung \(\tau\)).
Dann ist die Koinzidenzwahrscheinlichkeit an den Ausgängen
\[
P_{\text{coinc}}(\tau)=\frac{1}{2}\Bigl(1-|\Lambda(\tau)|^2\Bigr).
\]
Für perfekt überlappende, ununterscheidbare Photonen gilt \(|\Lambda(0)|=1\Rightarrow P_{\text{coinc}}(0)=0\).
Für zwei gaussförmige Wellenpakete mit Kohärenzzeit \(\tau_c\) ist
\(|\Lambda(\tau)|^2=\exp[-(\tau/\tau_c)^2]\),
sodass
\[
P_{\text{coinc}}(\tau)=\tfrac{1}{2}\Bigl(1-e^{-(\tau/\tau_c)^2}\Bigr)
\]
den charakteristischen \emph{HOM-Dip} zeigt.

\subsection*{Einfluss von Ununterscheidbarkeit.}
\phantomsection
Jede Unterscheidbarkeit (Polarisationswinkel \(\Delta\phi\), spektrale oder räumliche Modenfehler) reduziert die Sichtbarkeit \(V\in[0,1]\):
\[
P_{\text{coinc}}(\tau)=\frac{1}{2}\Bigl(1- V\,|\Lambda(\tau)|^2\Bigr),
\qquad
V=|\langle \xi_a|\xi_b\rangle|^2,
\]
wobei \(|\xi_{a,b}\rangle\) alle \emph{internen} Freiheitsgrade (z.\,B. Polarisation) beschreiben.

\subsection*{Bemerkung.}
\phantomsection
Das Verschwinden der Koinzidenzen ist kein klassischer Interferenzeffekt von Feldern, sondern eine \emph{Zwei-Photonen-Interferenz} der Wahrscheinlichkeitsamplituden\index{Wahrscheinlichkeitsamplitude}; es beweist die Ununterscheidbarkeit\index{Ununterscheidbarkeit von Photonen} und die bosonische Natur von
 Photonen.\index{Bosonen}
 
 
%\section{Kapitel V}
\section{Feldformalismus und Viererpotential}
\label{anhangA:feldformalismus}
\label{anhangA:viererpotential} % beide Labels zeigen auf dieselbe Stelle

In der Quantenelektrodynamik wird das Photon nicht als klassisches Teilchen,
sondern als Anregung des \emph{elektromagnetischen Feldes} beschrieben. 
Dieses Feld wird durch das \textbf{Viererpotential} \( A^\mu(x) \) dargestellt,
das in der relativistischen Formulierung vier Komponenten umfasst:

\[
A^\mu(x) = \big( \Phi(x), \, \vec{A}(x) \big) ,
\]

wobei \( \Phi(x) \) das elektrische Potential und \( \vec{A}(x) \) das magnetische Vektorpotential ist. 
Die zeitliche und räumliche Komponente vereinigen sich zu einem Lorentz-Vektor.

\subsection*{Feldstärketensor.}
\phantomsection
Aus dem Viererpotential erhält man den \textbf{Feldstärketensor}

\[
F_{\mu\nu} = \partial_\mu A_\nu - \partial_\nu A_\mu ,
\]

der die physikalischen Felder enthält:
\[
\vec{E} = -\nabla \Phi - \frac{\partial \vec{A}}{\partial t}, 
\quad
\vec{B} = \nabla \times \vec{A}.
\]

\subsection*{Lagrangedichte.}
\phantomsection
Die Dynamik des elektromagnetischen Feldes ergibt sich aus der
\textbf{Lagrangedichte}

\[
\mathcal{L}_{\text{EM}} = - \tfrac{1}{4} F_{\mu\nu} F^{\mu\nu} .
\]

Über das Prinzip der kleinsten Wirkung führt diese Formulierung auf die 
Maxwellschen Gleichungen in relativistischer Gestalt.

\subsection*{Eichsymmetrie.}
\phantomsection
Das Potential \( A^\mu \) ist nicht eindeutig bestimmt:
\[
A^\mu(x) \;\;\rightarrow\;\; A^\mu(x) + \partial^\mu \Lambda(x).
\]
Diese Freiheit heißt \textbf{Eichsymmetrie} und stellt sicher, dass nur
die physikalisch messbaren Größen \( \vec{E} \) und \( \vec{B} \) 
unabhängig von der Wahl des Potentials sind.

\medskip
Damit ist das Viererpotential die zentrale mathematische Struktur, aus der
sowohl die klassische Elektrodynamik als auch die quantisierte Form der QED
systematisch entwickelt werden können.
\section{Vom klassischen Feld zur QED}
\label{anhangA:feld_zu_qed}

Die klassische Elektrodynamik nach Maxwell beschreibt elektrische und 
magnetische Felder durch kontinuierliche Wellen, die sich im Raum ausbreiten. 
Das elektromagnetische Feld ist in dieser Sicht eine 
\emph{deterministische Lösung} der Maxwellschen Gleichungen. 

\subsection*{Grenzen des klassischen Modells.}
\phantomsection
Phänomene wie der Photoeffekt oder die Compton-Streuung zeigen jedoch,
dass Licht nicht beliebig teilbar ist, sondern nur in diskreten 
Energiepaketen (\( h\nu \)) mit Materie wechselwirkt. 
Dies deutet auf eine zugrunde liegende Quantennatur des Feldes hin.

\subsection*{Quantisierung des Feldes.}
\phantomsection
Die Quantenelektrodynamik (QED) geht über die klassische Theorie hinaus, 
indem sie das elektromagnetische Feld selbst \emph{quantisiert}. 
Das bedeutet:
\begin{itemize}
	\item Das Potential \( A^\mu(x) \) wird zum Operatorfeld.
	\item Seine Fourier-Moden entsprechen Erzeugungs- und Vernichtungsoperatoren 
	für Photonen.
	\item Die Zustände des Feldes werden im Fock-Raum beschrieben, 
	mit der Möglichkeit, beliebig viele Photonen in definierten 
	Moden zu erzeugen.
\end{itemize}

\subsection*{Neue Perspektive.}
\phantomsection
Das Photon erscheint somit als \textbf{Quant des elektromagnetischen Feldes}, 
nicht mehr als klassisches Teilchen oder Wellenpaket. 
Wechselwirkungen wie die Streuung zweier Elektronen werden als 
\emph{Photonenaustausch} verstanden, mathematisch dargestellt 
durch \textbf{Feynman-Diagramme}. 

\medskip
Damit schlägt die QED eine Brücke zwischen der klassischen Feldtheorie Maxwells, 
der Quantenmechanik und der speziellen Relativitätstheorie. 
Sie liefert ein konsistentes theoretisches Fundament, in dem das Photon 
als fundamentaler Austauschteilchen-Vektor bosonisch beschrieben wird.

\section{Feldstärketensor \(F_{\mu\nu}\)}

\label{anhangA:feldstaerketensor}

Die Grundlage der relativistischen Formulierung der Elektrodynamik
ist der \textbf{Feldstärketensor} \( F_{\mu\nu} \).
Er fasst die elektrischen und magnetischen Felder in einer 
kovarianten Form zusammen und wird direkt aus dem Viererpotential 
\( A^\mu(x) \) abgeleitet:

\[
F_{\mu\nu} \;=\; \partial_\mu A_\nu - \partial_\nu A_\mu .
\]

\subsection*{Eigenschaften.}
\phantomsection
\begin{itemize}
	\item \( F_{\mu\nu} \) ist \emph{antisymmetrisch}, d.\,h. 
	\( F_{\mu\nu} = - F_{\nu\mu} \).  
	\item Er enthält genau sechs unabhängige Komponenten, 
	die den drei Komponenten des elektrischen Feldes \( \vec{E} \) 
	und den drei Komponenten des magnetischen Feldes \( \vec{B} \) entsprechen.
\end{itemize}

\subsection*{Matrixdarstellung.}
\phantomsection
In 3+1-Darstellung ergibt sich:

\[
F_{\mu\nu} = 
\begin{pmatrix}
	0      & -E_x & -E_y & -E_z \\
	E_x    & 0    & -B_z & B_y \\
	E_y    & B_z  & 0    & -B_x \\
	E_z    & -B_y & B_x  & 0
\end{pmatrix}.
\]

\subsection*{Lorentz-Kovarianz.}
\phantomsection
Durch seine Definition ist \( F_{\mu\nu} \) ein Tensor zweiter Stufe
und transformiert sich wohldefiniert unter Lorentz-Transformationen.
Damit wird sichergestellt, dass elektrische und magnetische Felder
keine voneinander unabhängigen Größen sind, sondern je nach 
Beobachter ineinander übergehen.

\subsection*{Physikalische Bedeutung.}
\phantomsection
\begin{itemize}
	\item Der Feldstärketensor ist die zentrale Größe in der 
	Lagrangeformulierung der Elektrodynamik.
	\item Er erlaubt eine kompakte Darstellung der Maxwellschen Gleichungen.
	\item In der Quantenelektrodynamik bildet er die Grundlage 
	für die Definition der Photonenfelder und ihrer Wechselwirkungen.
\end{itemize}

\medskip
Damit fasst \( F_{\mu\nu} \) die klassischen Felder \( \vec{E} \) und \( \vec{B} \) 
in einer einheitlichen, relativistisch-invarianten Struktur zusammen.
\section{EM-Lagrangedichte und Gleichungen der Bewegung}
\label{anhangA:lagrange_em}

Die Dynamik des elektromagnetischen Feldes lässt sich elegant 
durch eine \textbf{Lagrangedichte} formulieren. 
Ausgangspunkt ist der Feldstärketensor \( F_{\mu\nu} \) 
(siehe Abschnitt~\ref{anhangA:feldstaerketensor}).

\subsection*{Lagrangedichte.}
\phantomsection
Die kanonische Form lautet
\[
\mathcal{L}_{\text{EM}} \;=\; -\tfrac{1}{4} \, F_{\mu\nu} F^{\mu\nu}.
\]

\begin{itemize}
	\item Der Vorfaktor \(-\tfrac{1}{4}\) ist notwendig, um bei der Variation 
	die richtigen Normierungen zu erhalten.
	\item Die kontrahierte Form \(F_{\mu\nu} F^{\mu\nu}\) 
	ist ein Lorentz-Skalar, also invariant unter Lorentztransformationen.
\end{itemize}

\subsection*{Kopplung an Materie.}
\phantomsection
Damit das Feld mit geladenen Teilchen wechselwirken kann,
fügt man einen Kopplungsterm ein:
\[
\mathcal{L}_{\text{int}} \;=\; - j_\mu A^\mu ,
\]
wobei \( j_\mu \) der Viererstrom ist.

\subsection*{Variation und Feldgleichungen.}
\phantomsection
Wendet man das \textbf{Prinzip der kleinsten Wirkung} auf
\[
\mathcal{L} \;=\; -\tfrac{1}{4} F_{\mu\nu}F^{\mu\nu} - j_\mu A^\mu
\]
an und variiert nach dem Potential \( A^\mu \), so erhält man:
\[
\partial_\nu F^{\mu\nu} \;=\; j^\mu .
\]

Dies sind die Maxwellschen Gleichungen in kompakter, 
kovarianter Form. Die inhomogenen Gleichungen
(\( \nabla \cdot \vec{E} = \rho, \, \nabla \times \vec{B} - \tfrac{\partial \vec{E}}{\partial t} = \vec{j} \))
sind darin enthalten.

\subsection*{Homogene Gleichungen.}
\phantomsection
Die beiden verbleibenden Maxwellschen Gleichungen 
(\( \nabla \cdot \vec{B} = 0, \, \nabla \times \vec{E} + \tfrac{\partial \vec{B}}{\partial t} = 0 \))
folgen aus der Definition des Feldstärketensors
und der Identität
\[
\partial_\lambda F_{\mu\nu} + \partial_\mu F_{\nu\lambda} + \partial_\nu F_{\lambda\mu} = 0 .
\]

\medskip
Damit zeigt sich, dass die gesamte klassische Elektrodynamik 
aus einer kompakten Lagrangeformulierung gewonnen werden kann – 
eine elegante Ausgangsbasis für die Quantisierung im Rahmen der QED.
\section{Eichsymmetrie, Eichfixierung und Lorenz-Bedingung}
\label{anhangA:eichsymmetrie}

Ein zentrales Strukturprinzip der Elektrodynamik ist die \textbf{Eichsymmetrie}.
Sie besagt, dass das Viererpotential \( A^\mu(x) \) nicht eindeutig bestimmt ist,
sondern bis auf eine sogenannte \emph{Eichtransformation}:

\[
A^\mu(x) \;\;\rightarrow\;\; A^\mu(x) + \partial^\mu \Lambda(x),
\]

wobei \( \Lambda(x) \) eine beliebige skalare Funktion ist.

\subsection*{Physikalische Konsequenz.}
\phantomsection
\begin{itemize}
	\item Die beobachtbaren Felder \( \vec{E} \) und \( \vec{B} \) 
	bleiben unter dieser Transformation unverändert.
	\item Nur eichinvariante Größen sind physikalisch messbar.
	\item Ein Masseterm \( \tfrac{1}{2} m^2 A_\mu A^\mu \) 
	wäre nicht eichinvariant und ist daher ausgeschlossen.
\end{itemize}

\subsection*{Eichfreiheit und Freiheitsgrade.}
Ein Vektorfeld \( A^\mu \) besitzt formal vier Komponenten.  
Die Eichsymmetrie erlaubt es jedoch, überflüssige Freiheitsgrade zu eliminieren:
\begin{itemize}
	\item Die Eichtransformation entfernt eine Komponente.
	\item Die Gleichungen der Bewegung (Lorentz-Invarianz) 
	eliminieren eine weitere.
	\item Es bleiben genau zwei unabhängige Freiheitsgrade – 
	die beiden transversalen Polarisationszustände des Photons.
\end{itemize}

\subsection*{Eichfixierung.}
\phantomsection
Um Rechnungen durchführen zu können, wählt man oft eine spezielle \emph{Eichung}:
\begin{itemize}
	\item \textbf{Lorenz-Eichung:} 
	\(\partial_\mu A^\mu = 0\).  
	Sie ist Lorentz-kovariant und besonders geeignet für relativistische Formulierungen.
	\item \textbf{Coulomb-Eichung:} 
	\(\nabla \cdot \vec{A} = 0\).  
	Sie wird häufig in der Quantenoptik verwendet.
\end{itemize}

\subsection*{Lorenz-Bedingung.}
\phantomsection
Die Lorenz-Eichung reduziert die Bewegungsgleichungen auf die Form einer 
Wellen- bzw. D’Alembert-Gleichung:
\[
\square A^\mu(x) = j^\mu(x),
\]
mit dem d’Alembert-Operator \(\square = \partial_\mu \partial^\mu\).
Dies macht die Wellennatur des elektromagnetischen Feldes deutlich.

\medskip
Die Eichsymmetrie ist somit nicht nur ein mathematisches Hilfsmittel, 
sondern der Grund, warum das Photon \textbf{masselos} ist und 
genau zwei transversale Polarisationszustände besitzt.
\section{Warum das Photon masselos ist  (Proca-Argument)}
\label{anhangA:masselosigkeit_proca}

In der klassischen Feldtheorie könnte man für ein Vektorfeld \( A^\mu \) 
formal einen Masseterm hinzufügen. Die entsprechende Lagrangedichte 
lautet dann (Proca-Theorie):

\[
\mathcal{L}_{\text{Proca}} = -\tfrac{1}{4} F_{\mu\nu} F^{\mu\nu} 
+ \tfrac{1}{2} m^2 A_\mu A^\mu .
\]

\subsection*{Folgen des Masseterms.}
\phantomsection
\begin{itemize}
	\item Die Gleichungen der Bewegung sind modifiziert und führen 
	auf eine massive Wellengleichung für das Feld.
	\item Ein massives Spin-1-Feld besitzt \textbf{drei} unabhängige 
	Polarisationszustände (statt zwei).
	\item Die Ausbreitungsgeschwindigkeit wäre kleiner als die Lichtgeschwindigkeit.
\end{itemize}

\subsection*{Eichsymmetrie verletzt.}
\phantomsection
Der Masseterm \( \tfrac{1}{2} m^2 A_\mu A^\mu \) 
ist nicht invariant unter der Eichtransformation
\[
A^\mu \;\to\; A^\mu + \partial^\mu \Lambda(x).
\]
Damit würde die fundamentale \(U(1)\)-Eichsymmetrie der Elektrodynamik gebrochen.

\subsection*{Experimentelle Evidenz.}
\phantomsection
Alle Beobachtungen zeigen, dass:
\begin{itemize}
	\item elektromagnetische Wellen sich stets mit Lichtgeschwindigkeit \(c\) ausbreiten,
	\item das Photon nur zwei transversale Polarisationszustände besitzt,
	\item und keine Abweichung von perfekter Masselosigkeit festgestellt wurde.
\end{itemize}
Daraus folgt, dass die Eichsymmetrie exakt gilt 
und das Photon eine exakt verschwindende Ruhemasse hat.

\medskip
Das \textbf{Proca-Argument} zeigt also:  
Die Eichsymmetrie der QED verbietet einen Masseterm – 
und zwingt das Photon, masselos zu sein.
\section{Transversalität und Helizität \( \pm 1 \)}
\label{anhangA:transversalitaet}

Das Photon ist ein masseloses Spin-1-Teilchen. Seine physikalisch
zulässigen Polarisationszustände ergeben sich aus der Kombination
von \textbf{Eichsymmetrie} und \textbf{Lorentz-Invarianz}.

\subsection*{Reduktion der Freiheitsgrade.}
\phantomsection
Ein Vektorfeld \( A^\mu \) hat zunächst vier Komponenten.
\begin{itemize}
	\item Die Eichfreiheit erlaubt es, eine Komponente durch Wahl 
	einer Eichung zu eliminieren.
	\item Die Bewegungsgleichungen (z.\,B. Lorenz-Bedingung 
	\( \partial_\mu A^\mu = 0 \)) entfernen eine weitere.
	\item Es bleiben genau \textbf{zwei unabhängige Freiheitsgrade}.
\end{itemize}

\subsection*{Transversalität.}
\phantomsection
Die beiden verbleibenden Polarisationsmoden sind 
senkrecht zur Ausbreitungsrichtung des Photons.  
Dies wird als \emph{Transversalität} bezeichnet:
\[
\vec{k} \cdot \vec{\epsilon}_\lambda = 0,
\]
wobei \( \vec{k} \) der Wellenvektor und 
\( \vec{\epsilon}_\lambda \) der Polarisationsvektor ist.

\subsection*{Helizität.}
\phantomsection
Für ein masseloses Teilchen wie das Photon ist die 
\textbf{Helizität} – die Projektion des Spins auf die 
Bewegungsrichtung –
\[
h = \frac{\vec{S} \cdot \vec{p}}{|\vec{p}|}
\]
eine wohldefinierte, Lorentz-invariante Größe.  
Das Photon besitzt genau zwei Helizitätszustände:
\[
h = +1 \quad \text{und} \quad h = -1 .
\]

\subsection*{Physikalische Interpretation.}
\phantomsection
\begin{itemize}
	\item Helizität \( +1 \): rechtszirkular polarisierte Photonen.
	\item Helizität \( -1 \): linkszirkular polarisierte Photonen.
\end{itemize}

\medskip
Damit folgt: Das Photon ist \textbf{transversal polarisiert}
und trägt nur zwei mögliche Helizitäten.  
Dies ist eine direkte Konsequenz der Eichsymmetrie und der
Masselosigkeit des Photons 
(vgl. Abschnitt~\ref{anhangA:masselosigkeit_proca}).
\section{Virtuelle Photonen und nichtphysikalische  Moden}
\label{anhangA:virtuelle_moden}

Während reale Photonen nur zwei transversale Helizitätszustände 
(\( h = \pm 1 \)) besitzen, treten in der Quantenelektrodynamik 
bei \textbf{virtuellen Photonen} zusätzliche Moden auf. 
Sie erscheinen in den inneren Linien von Feynman-Diagrammen und 
sind keine beobachtbaren Teilchen, sondern mathematische Hilfsgrößen 
im Formalismus.

\subsection*{Nichtphysikalische Komponenten.}
\phantomsection
In Propagatoren des Photonfeldes können neben den transversalen
auch longitudinale oder sogar skalare Anteile vorkommen.  
Diese entstehen aus der Eichfreiheit des Feldes 
und sind unvermeidlich, wenn man propagierende Lösungen in allen 
Komponenten zulässt.

\subsection*{Konsistenz der Theorie.}
\phantomsection
Obwohl solche nichtphysikalischen Moden in Zwischenrechnungen 
auftreten, verschwinden sie in allen \emph{physikalischen Observablen}.  
Dies geschieht durch:
\begin{itemize}
	\item die Eichinvarianz der Theorie,
	\item die Kopplung nur an den erhaltenen elektrischen Strom \( j^\mu \),
	\item und die Ward-Identitäten, die sicherstellen, 
	dass nichttransversale Beiträge herausfallen.
\end{itemize}

\subsection*{Beispiel: Photon-Propagator.}
\phantomsection
Im Feynman-Gauge lautet der Photon-Propagator
\[
D_{\mu\nu}(k) = \frac{-i g_{\mu\nu}}{k^2 + i\epsilon} ,
\]
der formal auch longitudinale und zeitartige Anteile enthält.  
In physikalischen Amplituden koppeln diese jedoch so,
dass sie sich exakt herauskürzen.

\medskip
\textbf{Fazit:}  
Virtuelle Photonen sind ein Recheninstrument der QED.  
Sie können scheinbar nichtphysikalische Moden enthalten, 
doch Eichsymmetrie und Stromerhaltung garantieren, 
dass nur die beiden transversalen Helizitätszustände des Photons 
physikalisch realisiert werden.
